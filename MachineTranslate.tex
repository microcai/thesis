\documentclass[a4paper]{article}
\usepackage[ urlcolor = blue, colorlinks = true, citecolor = black, linkcolor = black]{hyperref}
%\usepackage{graphicx}
%\usepackage{xltxtra}
\usepackage{fancyhdr}
\usepackage{booktabs}
\usepackage{indentfirst}
\usepackage{abstract}

\newcommand{\LABTP}{{Lexical Analyse Based Translation Program }}

\title{\LABTP}
\author{microcai}

\begin{document}

\maketitle

\begin{abstract}
\LABTP extends popular compiler model to support translating human languages.

\end{abstract}

\section{Introduction}

As moving to a new age, people tend to need more communication. Filling the gap
between different cultures requires translating countless books, articles and
words. Human resource is always a limitation. Thousands of pioneers have developed
a serious of Mathematical model for doing translation mechanically. 

Much of the previous model employ too much mathematics, focus not enough to the true meaning
of the words. Translate is not a simple mapping from source language to target language.
In order to do perfect translation, the true meaning of the words must be understood. 

\LABTP is based on the compiler model. What is a compiler? A translator between
one computer language to another. A compiler accept well defined computer language
text, and produce well defined computer language output. The computer language is well
defined, so it is possible to do lexical analyse. As for human languages? Hmm \ldots\ldots 
But, wait! Won't we really do lexical analyse on human language?

Well, It's partly possible. Think of English. 

\end{document}
