\documentclass[a4paper]{article}
\usepackage[utf8]{inputenc}
\usepackage[ urlcolor = blue, colorlinks = true, citecolor = black, linkcolor = black]{hyperref}
%\usepackage{graphicx}
%\usepackage{xltxtra}
\usepackage{fancyhdr}
\usepackage{booktabs}
\usepackage[cm-default]{fontspec}
\usepackage{makecell,rotating}
\usepackage[style=plaintop]{floatrow}
\usepackage{multirow}

\setlength{\parindent}{24pt}
 \setlength{\parskip}{3pt plus1pt minus2pt}
 \setlength{\baselineskip}{20pt plus2pt minus1pt}
 \setlength{\textheight}{21.5true cm}
 \setlength{\textwidth}{14.5true cm}
  \setlength{\headsep}{10truemm}
  \setlength{\oddsidemargin}{0.26cm}   % 左边 3.25cm=0.71+2.54
\setlength{\evensidemargin}{0.26cm}
%  上下
\setlength{\topmargin}{-0.29cm}   
 %\setlength{\jot}{4.5pt}


\newcommand\CJK{

%\setromanfont[Mapping=tex-text]{AR PL UMing CN}
\setmainfont{Adobe Song Std}
%\fontfamily{文泉驿等宽正黑}\selectfont{}

\XeTeXlinebreaklocale{zh-cn}
\XeTeXlinebreakskip=0em plus 0.1em minus 0.01em
\XeTeXlinebreakpenalty=0

} 

\newcommand\TimeNewRoman{

%\setromanfont[Mapping=tex-text]{AR PL UMing CN}
\setmainfont{Times New Roman}
%\fontfamily{文泉驿等宽正黑}\selectfont{}

\XeTeXlinebreaklocale{en-us}
\XeTeXlinebreakskip=0em plus 0.1em minus 0.01em
\XeTeXlinebreakpenalty=0

} 

%\usepackage{tableau}

\title{浙江理工大学外国语学院英语专业本科毕业论文开题报告}
\author{microcai}

\begin{document}

\begin{CJK}

\fontsize{15}{15.00}
\selectfont{}\textbf{浙江理工大学外国语学院英语专业本科毕业论文开题报告}

\end{CJK}

\begin{table}[h]
 \begin{CJK}
  \begin{tabular}{|m{55pt}|m{180pt}|m{55pt}|m{70pt}|}
     \hline
     \textbf{班级} & \makecell{小四号宋体加粗} & 姓名 & 小四号宋体加粗\\
     \hline
     课题名称 & \multicolumn{3}{l|}{ \textbf{浙江理工大学外国语学院英语专业本科毕业论文开题报告} } \\
     \hline
     \multicolumn{4}{|c|}{
       \makecell{      asasd  	}  } \\
     \hline
     成绩 & \multicolumn{3}{l|}{} \\
     \hline
	     \makecell{\textbf{答辩}\\\textbf{意见} } & \textbf{答辩组长签名} & 
	     \makecell{ \textbf{系}\\
	     	 \textbf{主}\\ \textbf{任}\\ \textbf{审}\\ \textbf{核}\\ \textbf{意}\\ \textbf{见} } &
	      \textbf{签名} \\
     \hline
  \end{tabular}
 \end{CJK} 
\end{table}

\part*{preface}

As moving to a new age, people tend to need more communication. Filling the gap between different cultures requires translating countless books, articles and words. Human resource is always a limitation. Thousands of 
pioneers have developed a serious of Mathematical model for doing translation mechanically. 

Much of the previous model employ too much mathematics, focus not enough to the true meaning of the words.
Translate is not a simple mapping from source language to target language. In order to do perfect translation,
the true meaning of the words must be understood. 


\end{document}
