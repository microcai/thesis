\documentclass[12pt,a4paper]{article}
\usepackage[ urlcolor = blue, colorlinks = true, citecolor = black, linkcolor = black]{hyperref}
\usepackage{fontspec}
\usepackage{graphicx}
%\usepackage{xltxtra}
\usepackage{fancyhdr}
\usepackage{booktabs}
\usepackage{makecell}
\usepackage{fancybox}

\usepackage{color}
\makeatletter
\renewcommand\tableofcontents{%
\section*{开题报告目录(全文附后)}%\contentsname
%\@mkboth{%
%\MakeUppercase\contentsname}{\MakeUppercase\contentsname}}%
{\color{black}%
\@starttoc{toc}}%
}\makeatother

%\usepackage[style=plaintop]{floatrow}

\setlength{\parindent}{24pt}
 \setlength{\parskip}{3pt plus1pt minus2pt}
 \setlength{\baselineskip}{20pt plus2pt minus1pt}
% \setlength{\textheight}{21.5true cm}
 \setlength{\textwidth}{15true cm}
  \setlength{\headsep}{10truemm}
  \setlength{\oddsidemargin}{0.26cm}   % 左边 3.25cm=0.71+2.54
\setlength{\evensidemargin}{0.26cm}
    
\newcommand\CJK{

\setmainfont{SimSun}
\setromanfont{SimSun}

\XeTeXlinebreaklocale{zh}
\XeTeXlinebreakskip=0pt plus 1pt
\XeTeXlinebreakpenalty=0

} 

\newcommand\TimeNewRoman{

%\setromanfont[Mapping=tex-text]{AR PL UMing CN}
\setmainfont{Times New Roman}
%\fontfamily{文泉驿等宽正黑}\selectfont{}

\XeTeXlinebreaklocale{en-us}
\XeTeXlinebreakskip=0em plus 0.1em minus 0.01em
\XeTeXlinebreakpenalty=0

} 

%\usepackage{tableau}

\title{浙江理工大学外国语学院英语专业本科毕业论文开题报告}
\author{microcai}

\begin{document}

\begin{CJK}
\fontsize{15}{15.00}
\selectfont{}\textbf{浙江理工大学外国语学院英语专业本科毕业论文开题报告}
\end{CJK}

\begin{table}[h]
 \begin{CJK}
  \begin{tabular}{|m{55pt}|m{180pt}|m{55pt}|m{80pt}|} 
  	  \hline
     \makecell{\textbf{班级}} & \makecell{\textbf{07英语(2)班}} & \makecell{\textbf{姓名}} & \makecell{\textbf{蔡万钊}}  \\
     \hline
     \textbf{课题名称} & \multicolumn{3}{c|}{ \textbf{基于语义分析的自动翻译}	} \\
     \hline

	\multicolumn{4}{|m{\textwidth-14pt}|}{
		\tableofcontents
	} \\       
    \hline          
    
    \textbf{成绩} & \multicolumn{3}{l|}{} \\
    \hline
    \makecell{\textbf{答辩}\\\textbf{意见} } & \textbf{答辩组长签名} & 
	     \makecell{ 
	     	\\\\\textbf{系}\\\textbf{主}\\\textbf{任}\\
	     	\textbf{审}\\\textbf{核}\\ \textbf{意}\\\textbf{见}\\\\
	      } & 	\multicolumn{1}{b{30pt}|}{ \textbf{签名}  }\\
     \hline
  \end{tabular}
 \end{CJK} 
\end{table}
\CJK{
\section{选题意义与可行性分析}
}

\begin{CJK}

进入新时代,人和人交流更多了。但是语言仍然是无阻碍交流里不可逾越的鸿沟,为了跨过这个鸿沟,我们花费了太多代价。
无数的书籍、文献亟待翻译。人力资源不仅短缺,也很昂贵。
无数先锋开发出了多种数学模型试图翻译,很多确实在工作的特别好。但是,要用一个数学模型去做翻译,缺点太多。
语言不是简单的能用公式就表达的清楚的。不是简单的用一个数学模型去隐射就能翻译的。
要正确的进行翻译,源文本的意思必须要被理解。

理解?如何让计算机理解文字?

这一方面编译器就做的很到位。高级语言本身已经很接近人类语言了,编译器依然能够理解并翻译成机器语言。
这种翻译是一种精确的翻译。人类的言语可以使用这样的模型翻译么?

可以,至少相当数量的句子是可以的。为何不去试一试呢?或者是,拿英语作为试验的对象。
和汉语不一样,英语具有非常严格的语法,每个句子都是非常完整的,很有逻辑。只有固定的几个句式,
每个句式的各个部分又可以嵌套进别的句式形成复合句型,非常适合用关系树去表达。

\end{CJK}

\CJK{
\section{研究的基本内容与拟解决的主要问题}
}

\begin{CJK}

本文旨在研究从英语生成语法树并对语法树进行汉语表达的可能性。

主要的问题是如何从一段英语文本生成抽象语法树,并使用合理的方式将语法树表达为汉语。

英语具有非常明显的结构,如果能将一个句子的各个单词的词性全部划分,那就非常容易构建语法树。
而为单词标记词性,已经是一个前人研究的结果了。

%TODO 查一下我最近看的到底是哪篇文章讲解了这一过程,加入文献引用库

\end{CJK}

\CJK{
\section{总体研究思路及预期研究成果}
}

\begin{CJK}

句子不是词语的堆砌。一个句子是有语法在里面的。 
如果完整的解析一个句子的意思,生成另一个中间语言。这个中间语言被叫做元语言。 在内存中可以用一个抽象语法树完整表达。
这个元语言必须是图灵完全而且是自描述的。 

然后再依据元语言,构造符合目标语言语法的,语义一致的一系列语句。 
接着使用统计模型,从中选择一条最有可能,正确的概率最大的作为最终的翻译结果。 

所以,机器翻译就被分解成几个子任务了。 

首先,需要构建一种元语言。这个语言应该是图灵完全和自描述的,并且可以在内存中用数据结构表示出来。由于我对英语最熟悉,所以我打算使用英语做为基础来构建元语言。 

第二,源语言到元语言的翻译。这个过程就是词法分析。找出每个单词的含义,并结合上下文和所处的环境,唯一确定下对应的元语言。如果含义无法唯一确定,可以构造多条元语言语句。待后续分析后再做决定 

第三,元语言的到目标语言的表达。这个时候必须依靠统计学,因为无法唯一确定元语言单词到目标语言单词的映射。需要基于统计学进行筛选。 

预期做出一个Demo完成翻译任务。

\end{CJK}

\CJK{
\section{研究工作计划}
}

\begin{CJK}
年底前拜读机器翻译大师的作品,找到一些前人已经有的功能,避免重复发明轮子。
次年开始构筑翻译Demo
次年3月完成Demo和论文。
\end{CJK}

\CJK{
\section{参考文献}
}

\CJK{
编译原理
Lex 和 Yacc
}

\end{document}
