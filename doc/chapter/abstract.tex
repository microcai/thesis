{
 \setCJKfamilyfont{song}[Mapping=tex-text,BoldFont=song-bold,ItalicFont=AR PL KaitiM GB]{Adobe Song Std}

\renewcommand{\abstractname}{摘要}
\renewcommand{\keywordsname}{关键词:}

\begin{abstract}
\CJKfamily{song}
许多工作,比如 \emph{机器翻译},\emph{人工语言处理(NLP)},需要对英文文本的真正意思进行理解。而为输入构筑语法树(CST,Concrete Syntax Tree, or Parse Tree\cite{cst})是``理解''的第一步。本论文叙述的就是为英语构筑语法树一种可能的方式。它利用了前人研究成果 \emph{基于隐马尔科夫模型(HMM)的语法标记}、\emph{基于巴科斯范式(BNF)的自然语言处理} 
构筑语法解析程序。

\keywords{语法树,机器翻译,隐马尔科夫模型,语法标记}
\end{abstract}

}

\clearpage

\begin{abstract}
Real understanding of English text is required by much of tasks such as \emph{Machine Translation}, \emph{NLP\footnote{Natural Language Processing}}, etc. Building Concrete Syntax \mbox{Tree \cite{cst}} is the very first step of the \mbox{understanding} process. This thesis brings you one possible way of building Parse Tree of Natural Language -- English. It employs \emph{HMM based part-of-speech tagging} and \emph{BNF\footnote{Backus–Naur Form\protect\cite{BNF}} based Natural Language Parse Program}. 

\keywords{Syntax Tree, Machine Translation, HMM, part-of-speech tagging}
\end{abstract}

\clearpage