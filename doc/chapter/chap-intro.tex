\section{Introduction}

\noindent As moving to a new age, people tend to need more communication.
Filling the gap between different cultures requires translating countless books, articles and words.
Human resource is always a limitation.
Thousands of pioneers have developed a serious of Mathematical model for doing translation mechanically. 

Much of the previous model employ too much mathematics, focus not enough to the true meaning
of the words.
Translate is not a simple mapping from source language to target language.
In order to do perfect translation, the true meaning of the words must be understood. 

So, what do we mean by \emph{understanding the meaning} ? 
Does a computer program can really understand a text ?
Think of a compiler\footnote{ A compiler translates one source computer language\footnotemark to another.}.\footnotetext{Usually programming language.}
Does the compiler (like gcc) understand your source code?
Of course yes.
Optimization\footnote{Compile can change the code so it runs faster but remains the same functionality. } can only occur when the compiler fully understands the source code.
What makes a compiler understand the code? It's AST --- Abstract Syntax Tree.
To make our program understand the English text, we need to build Concrete Syntax Tree for the English text. Concrete Syntax Tree is very similar to the Abstract Syntax Tree, we'll discuss them later.



%\LABTP is based on the compiler model. What is a compiler? A translator between
one computer language to another. A compiler accept well defined computer language
text, and produce well defined computer language output. The computer language is well
defined, so it is possible to do lexical analyse. As for human languages? Hmm \ldots\ldots 
But, wait! Won't we really do lexical analyse on human language?

Well, It's partly possible. Think of English. English is the mostly
NLP-friendly \footnote{NLP-friendly means, easy to do with Natural Language Processing} human language ever exist.
