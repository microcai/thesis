\section{Introduction}

\noindent As moving to a new age, people tend to need more communication.
Filling the gap between different cultures requires translating countless books, articles and words.
Human resource is always a limitation.
Thousands of pioneers have developed a serious of Mathematical model for doing translation mechanically. 

Much of the previous model employ too much mathematics, focus not enough to the true meaning
of the words.
Translate is not a simple mapping from source language to target language.
In order to do perfect translation, the true meaning of the words must be understood. 

So, what do we mean by \emph{ understanding the meaning} ? 
Does a computer program can really understand a text ?
Think of a compiler\footnote{ A compiler translates one source computer language\footnotemark to another.}.\footnotetext{ Usually programming language. }
Does the compiler (like gcc) understand your source code?
Of course yes.
Optimization\footnote{ Compile can change the code so it runs faster but remains the same functionality. } can only occur when the compiler fully understands the source code.
What makes a compiler understand the code? It's AST --- Abstract Syntax Tree.
To make our program understand the English text, we need to build Concrete Syntax Tree for the English text.
Concrete Syntax Tree is very similar to the Abstract Syntax Tree, we'll discuss them later.

When building parser tree, we need to know the grammar of the under-processing language, and tell this grammar to the parser program.
This can be done by constructing specific parser according to the language.
But we don't have to write that program piece by piece.
There's tool kit out there that can automatically do this as long as we provide the BNF\cite{BNF} declaration. We'll discuss BNF later.

In order  to let the automatically parser generating tool works, every word must be tagged with part-of-speech, because BNF declaration use the part-of-speech to distinguish sentence type.
And there is much way of determining this. We'll discuss them in chapter \ref{chap:HMM_based_tagging}.




