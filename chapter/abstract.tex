\thispagestyle{empty}%空白
{
\renewcommand{\abstractname}{摘要}
\renewcommand{\keywordsname}{关键词:}

\begin{abstract}
本文是浙江大学数{\bfseries 学系}学位论文的 \LaTeX{} 模板。除了介绍~\LaTeX{}~文档类
~\texttt{ZJUthesis}~的用法外,本文还是一个简要的学位论文写作指南。

\CJKfamily{song}本文是浙江大学数\textbf{学系}学位论文的 \LaTeX{} 模板。除了介绍~\LaTeX{}~文档类
~\texttt{ZJUthesis}~的用法外,本文还是一个简要的学位论文写作指南。

\keywords{浙江大学数学系,学位论文,\LaTeX{}~模板}
\end{abstract}

}

\clearpage

\thispagestyle{empty}%空白

\begin{abstract}
Real understanding of English text is required by much of tasks such as \emph{Machine Translation}, \emph{NLP\footnote{Natural Language Processing}}, etc. Building Concrete Syntax \mbox{Tree \cite{cst}} is the very first step of the \mbox{understanding} process. This thesis brings you one possible way of building Parse Tree of Natural Language -- English. It employs \emph{HMM based part-of-speech tagging} and \emph{BNF\footnote{Backus–Naur Form\protect\cite{BNF}} based Natural Language Parse Program}. 

\keywords{Zhejiang University , Thesis, \LaTeX{} Template}
\end{abstract}

\clearpage