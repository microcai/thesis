\section{Introduction}

As moving to a new age, people tend to need more communication. Filling the gap between different cultures requires translating countless books, articles and words. Human resource is always a limitation. Thousands of 
pioneers have developed a serious of Mathematical model for doing translation mechanically. 

Much of the previous model employ too much mathematics, focus not enough to the true meaning of the words.
Translate is not a simple mapping from source language to target language. In order to do perfect translation,
the true meaning of the words must be understood. 

Makeing a computer program understand a string is the so called Lexical Analyse. 
When Lexical Analyse finished, the given string will be converted into a abstract syntax tree. 
This tree represents the internal meaning and structure of the analysed string. Since it's a data struct, programs 
will find it more easier to do further analyse.\cite{Compilers_Principles_Techniques_and_Tools}

Befor doing the Lexcial Analyse, the string should be tokenized and the every token should be marked by it's part of speech.
This process is called, syntax analyse.

Tokenize can be performemed by regular expasion, mostly part of speech can be done by simply lookup in the dictionary. However

\section{Body}

\section{Conclusion}
\section{Bibliography}

\bibliography{../thesisbib}
\bibliographystyle{plain}
