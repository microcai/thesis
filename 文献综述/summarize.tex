\documentclass[12pt,a4paper]{article}
\usepackage[ urlcolor = blue, colorlinks = true, citecolor = black, linkcolor = black]{hyperref}
%\usepackage{graphicx}
%\usepackage{xltxtra}
\usepackage{fancyhdr}
\usepackage{booktabs}
\usepackage{CJK}
\usepackage{array}
\usepackage{makecell}
\usepackage{natbib}
\usepackage{setspace}
\usepackage{supertabular}
\usepackage{longtable}

\usepackage[bf, tiny]{titlesec}
    \titleformat{\section}{\bf}{\thesection{}. }{0.24em}{}
    \titlespacing{\section}{0cm}{*1.5}{*1.1}
    
\renewenvironment{thebibliography}[1]{%
      \list{\@biblabel{\@arabic\c@enumiv}}%
           {\settowidth\labelwidth{\@biblabel{#1}}%
            \leftmargin\labelwidth
            \advance\leftmargin\labelsep
            \@openbib@code
            \usecounter{enumiv}%
            \let\p@enumiv\@empty
            \renewcommand\theenumiv{\@arabic\c@enumiv}}%
      \sloppy
      \clubpenalty4000
      \@clubpenalty \clubpenalty
      \widowpenalty4000%
      \sfcode`\.\@m}
     {\def\@noitemerr
       {\@latex@warning{Empty `thebibliography' environment}}%
      \endlist}
\makeatother

%\setlength{\parindent}{24pt}
% \setlength{\parskip}{3pt plus1pt minus2pt}
% \setlength{\baselineskip}{20pt plus2pt minus1pt}
% \setlength{\textheight}{21.5true cm}
 \setlength{\textwidth}{15true cm}
%  \setlength{\headsep}{10truemm}
  \setlength{\oddsidemargin}{0.26cm}   % 左边 3.25cm=0.71+2.54
\setlength{\evensidemargin}{0.26cm}


\newcommand{\SONG}[1]{
{%
\begin{CJK}{UTF8}{gbsn}
#1%
\end{CJK}%
}}

\renewcommand{\arraystretch}{1.8}

\renewcommand{\contentsname}{ Contents (Full Text Appended)}

\title{Lexical Analyse Based Translation Program}
\author{microcai}

\begin{document}


\begin{CJK}{UTF8}{gbsn}
\fontsize{15.1}{15.1}
\textbf{浙江理工大学外国语学院英语专业本科毕业论文文献综述}\\
\end{CJK}


\begin{longtable}{|c|c|c|c|c|c|}
	\hline
	\endhead
	\hline
	\endfoot
	\hline
	\endfirsthead
	
     {\SONG{姓名}} & {\SONG{蔡万钊}} & {\SONG{班级}} &
       {07\SONG{英语}(2)\SONG{班}} & {{\SONG{指导老师}}} & {{\SONG{黄海军}}}  \\
     \hline
     \SONG{论文题目} & \multicolumn{5}{c|}{\textbf{\SONG{基于语义分析的自动翻译}}} \\
     \hline 
     {Thesis Title} & \multicolumn{5}{c|}{\textbf{Lexical Analyse Based Translation Program}} \\    
     \hline

	\multicolumn{6}{|m{420pt}|}{ 
	
	\begin{onehalfspace}
\section{Introduction}

As moving to a new age, people tend to need more communication. Filling the gap between different cultures requires translating countless books, articles and words. Human resource is always a limitation. Thousands of 
pioneers have developed a serious of Mathematical model for doing translation mechanically. 

Much of the previous model employ too much mathematics, focus not enough to the true meaning of the words.
Translate is not a simple mapping from source language to target language. In order to do perfect translation,
the true meaning of the words must be understood. 


\section{Body}
\section{Conclusion}
\section{Bibliography}
\end{onehalfspace}	

	 } \\  
	
	\newpage

	\multicolumn{6}{|c|}{} \\
    \multicolumn{1}{|l}{\SONG{成绩}} & \multicolumn{5}{l|}{} \\
    
	\multicolumn{6}{|c|}{} \\
	\multicolumn{6}{|c|}{} \\
		
	\multicolumn{2}{|l}{\SONG{指导教师审阅意见} } 	& \multicolumn{4}{l|}{}  \\

	\multicolumn{6}{|l|}{} \\

	\multicolumn{4}{|c}{} & \multicolumn{2}{l|}{\SONG{指导教师签名\underline{\qquad\qquad\qquad}}} \\

	\multicolumn{6}{|r|}{ \SONG{年}\quad \SONG{月}\quad \SONG{日}} \\
	
    \hline
\end{longtable}

\end{document}
