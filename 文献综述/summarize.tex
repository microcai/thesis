\documentclass[12pt,a4paper]{article}
\usepackage[ urlcolor = blue, colorlinks = true, citecolor = black, linkcolor = black]{hyperref}
%\usepackage{graphicx}
%\usepackage{xltxtra}
\usepackage{fancyhdr}
\usepackage{booktabs}
\usepackage{indentfirst}
\usepackage{CJK}
 

\title{Lexical Analyse Based Translation Program}
\author{microcai}

\begin{document}

\part*{Introduction}

As moving to a new age, people tend to need more communication. Filling the gap between different cultures requires translating countless books, articles and words. Human resource is always a limitation. Thousands of 
pioneers have developed a serious of Mathematical model for doing translation mechanically. 

Much of the previous model employ too much mathematics, focus not enough to the true meaning of the words.
Translate is not a simple mapping from source language to target language. In order to do perfect translation,
the true meaning of the words must be understood. 

\begin{CJK}{UTF8}{gbsn}
的
\end{CJK}


\end{document}
